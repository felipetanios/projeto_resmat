\documentclass[a4paper]{article}

% algumas packages para arrumar as tables co a margin:
% allows for temporary adjustment of side margins
\usepackage{changepage}
% provides filler text
\usepackage{lipsum}

% just makes the table prettier (see \toprule, \bottomrule, etc. commands below)
\usepackage{booktabs}


\usepackage[brazilian]{babel} %Para traduzir os textos
\usepackage[utf8]{inputenc} %Para poder usar acentos
\usepackage[a4paper]{geometry} %Para ajustar a parte geometrica da folha
\geometry{verbose,tmargin=2cm,bmargin=3cm,lmargin=3cm,rmargin=3cm} %Parte de margens
\setlength{\parindent}{0.5cm}
\usepackage{wrapfig} %Biblioteca Matematica/Grafica
\usepackage{mathptmx} %Biblioteca Matematica/Grafica
\renewcommand{\ttdefault}{mathptmx} %Biblioteca Matematica/Grafica
\usepackage{amsmath} %Biblioteca Matematica/Grafica
\usepackage{amssymb} %Biblioteca Matematica/Grafica
\usepackage[11pt]{moresize}% different letters sizes
\usepackage{float}% enables accurate location of tables
\usepackage{caption}% to make personalized captions
\usepackage{graphicx} %Para inclusão de imagens
\usepackage{indentfirst}
\usepackage{amsfonts}
\usepackage[T1]{fontenc}
%\usepackage[nottoc]{tocbibind}
\usepackage{mcode}


\makeatletter
\providecommand{\tabularnewline}{\\} %define

\title{Cálculo Computacional do Círculo de Mohr} % main title
\author{EM423 - Resistencia dos Materiais \\ \textsc{Grupo 13}}
\date{22 de Outubro, 2014}

\begin{document} % actually starts the document here
\maketitle

% members of the group
\begin{center}

	\begin{tabular}{l r l}
		Integrantes:\\\\
		 Felipe Tanios & RA: 155330  \\
		 Henrique Noronha Facioli & RA: 157986 \\
		 Kairo Vinicius & RA: 156075 \\
		 Lauro Souza e Cruz & RA: 156175\\
		 Mateus Coelho & RA: 156675 \\
	\end{tabular}
\end{center}


\newpage

\section{Introdução}


A força é uma ação que tende a alterar o estado do corpo,seja em repouso ou em movimento. É possível fazer o corpo acelerar, modificar a velocidade, a direção e o sentido de movimento. 
O estado do corpo submetido à ação de forças em torno de um ponto dentro do corpo material, é chamado de tensão. Força de tensão é uma força que quando aplicada a um corpo elástico, ele tende a se modificar, produzindo uma tensão. A representação desse estado de tensões é chamado de Círculo de Mohr.

Introduzido por  Christian Otto Mohr,  o Círculo de Mohr é uma forma gráfica bidimensional para resolver problemas de momentos de inércia, deformações e estado de tensões baseado na lei de transformação do tensor tensão de Cauchy. É possível usar o Círculo de Mohr somente quando cada plano for representado por um ponto em um sistema de coordenadas ($\sigma,\tau$).

O circulo de Mohr foi muito usado no passado quando não existiam calculadoras eletrônicas para obter graficamente e em escala, respostas para os problemas de distribuição de Tensões, porem atualmente ele é mais usado para visualizar completamente os estados de tensão em um ponto P considerado

\section{Objetivo}
Prover uma interface de fácil utilização que, dadas as tensões de cisalhamento e as tensões normais em um plano, exibe em um gráfico personalizado o círculo de Mohr. O programa pode ser executado em múltiplas plataformas (Windows, Linux) e gera um log em formato de página de internet com os dados calculados. 

\section{Codigo}


\section{Formulas}

\begin{equation}
	\label{sigma_x'}
	\sigma_{x'} = \frac{\sigma_x + \sigma_y}{2} + \frac{\sigma_x - \sigma_y}{2}cos(2 \theta) + \tau_{xy}sen(2\theta)
\end{equation}

\begin{equation}
	\label{tau_xy'}	
	\tau_{x'y'} = -\frac{\sigma_x - \sigma_y}{2}sen(2 \theta) + \tau_{xy}cos(2\theta)
\end{equation}

\begin{equation}
	\label{sigma_y'}
	\sigma_{y'} = \frac{\sigma_x + \sigma_y}{2} - \frac{\sigma_x - \sigma_y}{2}cos(2 \theta) - \tau_{xy}sen(2\theta)
\end{equation}

\begin{equation}
	\sigma_x + \sigma_y = \sigma_{x'} + \sigma_{y'}
\end{equation}

\begin{equation}
	(\sigma_{x'} - \frac{\sigma_x + \sigma_y}{2})^2 + \tau_{x'y'}^2 = (\frac{\sigma_x - \sigma_y}{2})^2 + \tau_{xy}^2
\end{equation}

\begin{equation}
	\sigma_{med} = \frac{\sigma_x + \sigma_y}{2} 
\end{equation}

\begin{equation}
	R =  \sqrt{(\frac{\sigma_x - \sigma_y}{2})^2 + \tau_{xy}^2}  
\end{equation}

\begin{equation}
	tg(2\theta_p) = \frac{2\tau_xy}{\sigma_x - \sigma_y}
\end{equation}

\begin{equation}
	\sigma_{max} = \sigma{med} + R \therefore \sigma_{max} = \frac{\sigma_x + \sigma_y}{2} + \sqrt{(\frac{\sigma_x - \sigma_y}{2})^2 + \tau_{xy}^2}
\end{equation}

\begin{equation}
	\sigma_{max} = \sigma{med} + R
\end{equation}

\begin{equation}
	\tau_{max} = \sqrt{(\frac{\sigma_x - \sigma_y}{2})^2 + \tau_{xy}^2}
\end{equation}

\begin{equation}
	tg(2\theta_s) = - \frac{\sigma_x - \sigma_y}{2\tau_{xy}}
\end{equation}


\section{Bibliografia}

\bibliography{mybib}


\end{document}
